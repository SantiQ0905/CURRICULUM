\documentclass[letterpaper,11pt]{article}

%-------------------------  Package Imports -------------------------%
\usepackage{latexsym}
\usepackage[empty]{fullpage}
\usepackage{titlesec}
\usepackage{marvosym}
\usepackage[usenames,dvipsnames]{color}
\usepackage{verbatim}
\usepackage{enumitem}
\usepackage[hidelinks]{hyperref}
\usepackage{fancyhdr}
\usepackage[english]{babel}
\usepackage{tabularx}
\input{glyphtounicode}

%-------------------------  Page Styles -------------------------%
\pagestyle{fancy}
\fancyhf{} 
\fancyfoot{}
\renewcommand{\headrulewidth}{0pt}
\renewcommand{\footrulewidth}{0pt}

%-------------------------  Margin Adjustments -------------------------%
\addtolength{\oddsidemargin}{-0.5in}
\addtolength{\evensidemargin}{-0.5in}
\addtolength{\textwidth}{1in}
\addtolength{\topmargin}{-.5in}
\addtolength{\textheight}{1.0in}

%-------------------------  Alignment commands -------------------------%
\urlstyle{same}
\raggedbottom
\raggedright
\setlength{\tabcolsep}{0in}

%-------------------------  Sections Formating -------------------------%
\titleformat{\section}{
  \vspace{-4pt}\scshape\raggedright\large
}{}{0em}{}[\color{black}\titlerule \vspace{-5pt}]

%-------------------------  ATS Readable -------------------------%
\pdfgentounicode=1

%-------------------------  Custom commands -------------------------%

    \newcommand{\resumeItem}[1]{
      \item\small{
        {#1 \vspace{-2pt}}
      }
    }
    
    \newcommand{\resumeSubheading}[4]{
      \vspace{-2pt}\item
        \begin{tabular*}{0.97\textwidth}[t]{l@{\extracolsep{\fill}}r}
          \textbf{#1} & #2 \\
          \textit{\small#3} & \textit{\small #4} \\
        \end{tabular*}\vspace{-7pt}
    }
    
    \newcommand{\resumeSubSubheading}[2]{
        \item
        \begin{tabular*}{0.97\textwidth}{l@{\extracolsep{\fill}}r}
          \textit{\small#1} & \textit{\small #2} \\
        \end{tabular*}\vspace{-7pt}
    }
    
    \newcommand{\resumeProjectHeading}[2]{
        \item
        \begin{tabular*}{0.97\textwidth}{l@{\extracolsep{\fill}}r}
          \small#1 & #2 \\
        \end{tabular*}\vspace{-7pt}
    }
    
    \newcommand{\resumeSubItem}[1]{\resumeItem{#1}\vspace{-4pt}}
    
    \renewcommand\labelitemii{$\vcenter{\hbox{\tiny$\bullet$}}$}
    
    \newcommand{\resumeSubHeadingListStart}{\begin{itemize}[leftmargin=0.15in, label={}]}
    \newcommand{\resumeSubHeadingListEnd}{\end{itemize}}
    \newcommand{\resumeItemListStart}{\begin{itemize}}
    \newcommand{\resumeItemListEnd}{\end{itemize}\vspace{-5pt}}

%-------------------------  Custom commands -------------------------%

%-------------------------  RESUME STARTS HERE -------------------------%
\begin{document}

\begin{center}
	\textbf{\Huge \scshape Santiago Quintana Moreno} \\ \vspace{1pt}
	\small +52 5538890890 $|$ \href{mailto:santy.10qm.gm@gmail.com}{\underline{santy.10qm.gm@gmail.com}} $|$
	\href{https://github.com/SantiQ0905}{\underline{https://github.com/SantiQ0905}} $|$
        \href{https://twitter.com/10qmSanty}{\underline{https://twitter.com/10qmSanty}} $|$
        \href{https://www.santiagoquintanamoreno.com/}{\underline{https://www.santiagoquintanamoreno.com/}} 
 
\end{center}


%-----------EDUCATION-----------%
\section{Educación}
\resumeSubHeadingListStart
\resumeSubheading
{West Vancouver Secondary School}{Vancouver, British Columbia, Canadá}
{Bachillerato 2 semestres.}{Agosto 2018 -- Junio 2019}
\resumeSubHeadingListEnd

\resumeSubHeadingListStart
\resumeSubheading
{PrepaTec}{Monterrey, Nuevo León, México}
{Bachillerato 6 semestres.}{Agosto 2019 -- Mayo 2022}
\resumeSubHeadingListEnd

\resumeSubHeadingListStart
\resumeSubheading
{Münchner Volkshochschule}{München, Bavaria, Alemania}
{Habilidades de Alemán avanzado.}{Enero 2023 -- Marzo 2023}
\resumeSubHeadingListEnd

\resumeSubHeadingListStart
\resumeSubheading
{Tecnológico de Monterrey}{Monterrey, Nuevo León, México}
{Ingeniería en Tecnologías Computacionales (ITC)}
{Agosto 2023 -- Presente}
\resumeSubHeadingListEnd
%-----------EDUCATION-----------%

%-----------EXPERIENCE-----------%
\section{Experiencia}
\resumeSubHeadingListStart

\resumeSubheading
{Mentor de Robótica}{Mayo 2022 -- Presente}
{PrepaTec Overture 7421, 23550, 23619}{Monterrey, México}
\resumeItemListStart
\resumeItem{Instruyo a los miembros del equipo en cableado, electronicos, C++, Java, Python y WPILib, lo que facilitó a mis alumnos generar una programación avanzada de robots de competencia para una perfecta integración de hardware y software.}
\resumeItem{Instruyo sobre la recopilación de datos a través del desarrollo y análisis de aplicaciones de scouting e inteligencia de robots internas.}
\resumeItem{Colaboración guiada en el equipo a lo largo de las temporadas, optimizando tareas y recursos para un rápido y eficiente diseño y fabricación de robots de competición. Al mismo tiempo, brindo liderazgo y tutoría durante los eventos, cultivando la toma de decisiones estratégicas, el trabajo en equipo y la adaptabilidad.}
\resumeItem{Colaboré en el desarrollo, la construcción y el primer año del equipo FLL 23550.}
\resumeItem{Colaboré en el desarrollo, la construcción y obtuve el estatus de Mentor principal en programación, scouting y equipo de conducción para el equipo FTC 23619.}
\resumeItem{Nominado al premio Wodie Flowers Mentorship en el Regional Monterrey presentado por PrepaTec en 2024.}
\resumeItemListEnd

\resumeSubHeadingListEnd


%-----------PROJECTS-----------%
\section{Proyectos}
\resumeSubHeadingListStart
\resumeProjectHeading
{\textbf{Smart Breathing} $|$ \emph{CAD, Python, Arduino, Github}}{2022 - 2023}
\resumeItemListStart
\resumeItem{Nominado al “Premio Eugenio Garza Sada” en la categoría de Mejor Proyecto de Innovación Social Estudiantil por la creación y desarrollo de “Smart Breathing”. Junto con mis compañeros, desarrollamos y creamos prototipos de dos dispositivos deportivos diseñados para la detección y alertas de contaminación. Además, creamos una mascarilla transpirable diseñada para filtrar eficazmente las partículas contaminantes, que también brindó asistencia durante la pandemia de COVID-19.}
\resumeItemListEnd

\resumeProjectHeading
{\textbf{ECOCYCLE, Xignux Challenge} $|$ \emph{CAD, Python, Arduino, Github, Notion}}{2024 - Presente}
\resumeItemListStart
\resumeItem{Colaboré en el diseño de una trituradora y extrusora de plástico sostenible y replicable. Esta máquina procesa plástico reciclado en formas utilizables, con doble aplicación: emprendimiento e impacto social. Esta innovadora máquina puede procesar plástico reciclado en formas utilizables para diferentes propósitos. Para lograr un impacto social, extruye láminas de plástico diseñadas para construir paredes en viviendas de bajos ingresos, proporcionando un material de construcción rentable y ecológico. En cuanto al emprendimiento, la máquina apoya un negocio centrado en la creación de arte para paredes, pisos y techos a partir de plástico reciclado, ofreciendo una solución asequible y creativa para decorar espacios.}
\resumeItemListEnd

\resumeSubHeadingListEnd

%-----------Technical Skills-----------%
\section{Habilidades Técnicas}
\begin{itemize}[leftmargin=0.15in, label={}]
	\small{\item{
		            \textbf{Lenguajes de Programación}{: Java, Python, C/C++, HTML/CSS, LaTeX} \\
		            \textbf{Herramientas de Desarrollo}{: Git, Github, VS Code, Arduino IDE, Android Studio, Jupyter, Matlab} \\
		            \textbf{Librerías}{: NumPy, Matplotlib, mpltoolkits, mplot3d, Axes3D, WPILib, FTCLib, RoadRunner} \\
                        \textbf{Habilidades Sociales}{: Adaptabilidad, Comunicación efectiva y puntual, Toma de decisiones, Resolución de problemas, Trabajo en equipo} \\
                        \textbf{Software General}{: Microsoft Office, GSuite, Notion, Slack} \\
		            \textbf{Idiomas}{: Español (Nativo), Inglés (C1), Alemán (B2)} \\
		      }}
\end{itemize}
%-----------Technical Skills-----------%

\end{document}